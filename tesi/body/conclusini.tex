\chapter{Conclusioni}
In conclusione lo stage non si è concentrato su una singola parte in maniera dettagliata e approfondita, bensì su una ampia gamma di strumenti e linguaggi.\\
Ho lavorato sia allo sviluppo del frontend in React, sia allo sviluppo del backend in Node e Python ma oltre lo studio di nuovi linguaggi, framework e paradigmi di programmazione la parte che ho trovato più formativa (e stimolante) è stato impostare tutto il sistema per poter distribuire l'applicazione. Non ho trattato questa parte nella tesi poichè non la ritengo particolarmente interessante e ci ho dedicato solo una piccola parte del tempo (non è complicato o lungo settare un server web), eppure mi ha costretto a lavorare con strumenti che non conoscevo e a studiare personalmente una soluzione ottimale per poter far funzionare tutto al meglio.\\

\noindent
Un altro punnto che voglio trattare è il presente e il futuro dell'applicazione. Attualmente siamo andati avanti nello sviluppo da ciò che ho inserito nello stage. Per quanto riguarda la società abbiamo in programma, io e il mio socio, di costituirci come startup (srl innovativa) a breve, una volta costituiti affideremo lo sviluppo frontend mancante a una socità estenra per accellerare il processo di sviluppo. A livello di backend sto finendo di sviluppare l'api Node con la parte riguardante l'acquisto e la vendita di portafogli. L'obiettivo attuale è riuscire a lanciare per inizio gennaio 2022 una beta su un campione di medie dimensioni di studenti per poter acquisire importanti dati su utilizzo e usability che ci daranno una corretta indicazione sulla direzione che dovrebbe prendere lo sviluppo e ci permetteranno di definire ulteriormente il modello di business. 

