\chapter{Introduzione}



\section{Introduzione sul progetto WITZ}
WITZ nasce dall’idea del mio socio Matteo Pellandino di tradurre in algoritmo e implementare una teoria finanziaria studiata nel suo corso di studio. Questa teoria è chiamata "modello di Markowitz" e viene utilizzata per la costruzione di portafogli finanziari.\\
Siamo partiti quindi dalla costruzione di una prima rudimentale libreria che implementa la teoria di base, cioè che calcolava la percentuale del capitale da investire in ogni titolo presente in un dato portafoglio, utilizzando come input dell'utente un certo rendimento.\\
Questa libreria è stato il punto di partenza di tutto il lavoro, infatti abbiamo lavorato per ampliarla e migliorarla fino ad arrivare a una libreria contenente varie funzioni per calcolare un portafoglio di ottimo partendo da input diversi o richiedendo tipi di portafoglio diversi.\\
In parallelo abbiamo portato avanti lo sviluppo di WITZ, una applicazione web e mobile che offrirà sia la possibilità di investire sia la possibilità di accedere a dei contenuti di educazione finanziaria, con la mission di provare a migliorare il livello, molto basso, di competenze finanziarie dei giovani italiani e aiutarli a risparmiare e investire in maniera consapevole. 

\section{I giovani italiani non conoscono la finanza}
Una ricerca pubblicata nel 2018 dalla Banca d’Italia ha rilevato un gap sostanziale fra il nostro paese e il resto dell’area Ocse quanto al livello di conoscenze di base dei temi legati alla finanza personale, al risparmio e agli investimenti: soltanto il 30\% degli italiani ha raggiunto un livello di conoscenza adeguato di questi aspetti, contro una media Ocse del 62\%.\\
L’analfabetizzazione finanziaria è diffusa soprattutto tra i giovanissimi, i quali hanno una buona propensione a risparmiare, ma pochissima inclinazione ad investire. Ciò è dovuto principalmente al fatto che gli studenti non bancarizzati in Italia sono moltissimi (circa il 65\% contro una media OCSE del 44\%, dato che diventa il 19\% riferito a tutte le età contro una media EU15 del 10\%) e ciò genera una forte barriera all’entrata dei mercati finanziari (non avendo conoscenza adeguata e non ricevendo proposte di impiego del capitale dagli intermediari finanziari).\\
Per quanto riguarda invece gli individui che investono i propri risparmi, la maggior parte di questi, non avendo gli strumenti necessari per costruire un portafoglio ottimale, si affida ad intermediari finanziari capaci di garantire un’efficace diversificazione del rischio ed una strategia competitiva. Questo però in molti casi genera elevati costi di gestione e conflitti di interesse tra le parti.\\
Un’altra categoria che riscontra bias comportamentali dovuti alla scarsa conoscenza finanziaria è quella dei trader al dettaglio. Si stima che il 75\% dei conti correnti sulle piattaforme digitali sia in perdita. Questo fatto è dovuto in gran parte all’impossibilità da parte dei privati di costruire un portafoglio di investimenti che rispecchi le loro esigenze ed alla mancanza di una strategia ideale da applicare ai mercati finanziari.


\section{La teoria finanziaria di Makrowitz in breve}

Harry Markowitz (Chicago, 24 agosto 1927) è un economista statunitense, e teorizzò la “teoria del portafoglio di Markowitz”. Questa teoria si basa sul trovare una correlazione tra i titoli, ovvero le azioni, presenti in un insieme, detto portafoglio finanziario. Questa correlazione si trova tramite calcoli matriciali e gli input iniziali sono una lista di titoli e i relativi rendimenti storici.\\
Questi rendimenti storici vengono utilizzati per costruire due oggetti: una matrice contenente le covarianze tra i titoli, che rappresenta la correlazione tra i titoli, e una lista di rendimenti medi, che vengono utilizzati come rendimenti attesi, ovvero quanto ci si aspetta che il titolo in media potrà rendere in un certo periodo in avanti nel futuro. Sostanzialmente per ottenere i rendimenti medi si analizza un periodo passato, supponendo che sia simile al periodo nel futuro in cui si vuole investire. \\
Una volta trovata la correlazione tra i titoli è possibile bilanciare correttamente il portafoglio in modo che i titoli si compensino l’un l’altro. Si ha correlazione tra due titoli, semplificando la realtà, se quando il titolo X sale, il titolo Y, correlato al titolo X, scende. L’effetto globale ottenuto è che inserendo questi due titoli in un portafoglio quando vado a perdere con uno, poiché sta scendendo e quindi perdendo valore, l’altro a lui correlato sarà in crescita, controbilanciando la perdita.\\
Partendo dalla teoria e l’algoritmo di base è poi possibile costruire altri strumenti. Il teorema di Markowitz costruisce una curva, nell’algoritmo di base viene richiesto un certo rendimento, che corrisponde a una certa y, si va quindi a ricercare sulla curva il punto corrispondente a quella y.\\ 
Da qui poi altri strumenti costruiti potrebbero al posto di richiedere il rendimento (la y) potrebbero richiedere il rischio (la x), modificando leggermente la formula. Altra opzione ancora è il calcolo del portafoglio inserendo un “titolo certo”, ovvero un titolo generalmente con poco rendimento ma che ha anche un rischio molto basso, viene quindi utilizzato come “salvagente” poiché assicura un minimo di rendimento. Nel calcolo del portafoglio di ottimo (ovvero il portafoglio che si ottiene) è possibile anche inserire con una modifica della formula iniziale anche il titolo certo. 


\section{Cosa ho fatto nello stage?}

Il mio stage (più il prolungamento) si è concentrato su due aspetti: la costruzione di parte della libreria e lo sviluppo dell’applicazione. 
La parte dello stage riguardante lo sviluppo della libreria è a sua volta suddivisa. Uno dei compiti è stato la trascrizione in algoritmo della procedura “a mano” per il calcolo del portafoglio di ottimo, con annesso tutto ciò che era necessario al calcolo, come funzioni ausiliarie ai calcoli matriciali, funzioni per il recupero dei dati utilizzando api e la strutturazione di svariati test per verificare sia il funzionamento dell’algoritmo, sia l’effettiva efficacia del metodo, poiché una volta certi che funzionasse volevamo verificare che funzionasse bene. 
La parte riguardante o sviluppo dell’app è partita da una iniziale fase di progettazione, scelta dei linguaggi/framework e il loro apprendimento. Una volta scelte le tecnologie e aver abbozzato le funzionalità base ho iniziato a svilupparla. Lo sviluppo è stato incrementale poiché man mano che andavo avanti riscontravamo la necessità di inserire funzioni o cambiare il layout grafico, che è stato implementato da me, ma studiato assieme alla graphic designer del nostro team. Parte dello sviluppo ha anche incluso lo studio di una soluzione che permettesse la distribuzione, quindi il corretto funzionamento di api e app su un ip pubblico.