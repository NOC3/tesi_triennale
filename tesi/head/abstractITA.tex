\chapter*{Abstract}

Lo stage si è concentrato sullo sviluppo di WITZ, webapp finanziaria progettata dal sottoscritto in team con uno studente di finanza.
L'obbiettivo di WITZ è avvicinare i giovani alla finanza in maniera consapevole, offrendo sia un servizio di educazione finanziaria, sia un servizio di risparmio e investimento consapevole, basato sul livello di competenza dell'utente. 
L'obiettivo dello stage è stato da una parte sviluppare  il modulo di educazione finanziaria con l'obbiettivo di ottenere un prodotto funzionante e con una gradevole user experience, dall'altra progettare e implementare gli algoritmi necessari ad applicare la teoria finanziaria della costruzione dei portafogli di Markowitz, realizzando  una libreria che permetta l'integrazione con l'applicazione e il testing.
Con portafoglio di investimento si intende una certa combinazione di titoli finanziari con i relativi pesi. Il cliente investe una percentuale del proprio capitale su un certo titolo, percentuale pari al peso di questo titolo nel portafoglio. Così facendo avrà una alta probabilità, legata al livello di rischiosità del portafoglio, di ottenere i risultati calcolati tramite la teoria di Markowitz. 
L'algoritmo di base è stato arricchito con diverse varianti, per permettere all'utente di calcolare vari tipi di portafogli, ad esempio decidendo se volesse partire dal rendimento atteso, dal rischio collegato al portafoglio oppure avere un portafoglio senza vendita allo scoperto.
l frontend dell'applicazione è sviluppato in React JS e si appoggia a due servizi esposti online: il primo, realizzato in Node.JS, fornisce i dati riguardanti le lezioni e i progressi dell'utente, il secondo è scritto in python e realizza la componente algoritmica e di reperimento dati. 