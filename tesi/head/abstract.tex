\chapter*{English Abstract}

The intership focused on develop WITZ, financial WebApp designed by me in team with a finance student.
WITZ's goal is approach young people to finance in a conscious way, offering both a 
financial education service, both a saving and investment service aware based on the level of user competence.
The aim of the internship was on the one hand to develop the financial education module with the aim of obtaining a working product and with a pleasant user experience, on the other hand, design and implement the necessary algorithms to apply the financial theory of the construction of Markowitz wallets, creating a library that allows integration with the application and testing.
Investment wallet means a certain combination of financial securities with their weights. The customer invests a percentage of its capital on a certain security, a percentage equal to the weight of this security in the wallet. By doing so it will have a high probability, linked to the level of riskiness of the wallet, of obtaining the results calculated by Markowitz’s theory.
The basic algorithm has been enriched with several variants, to allow the user to calculate various types of wallets, for example deciding whether he wanted to start from the expected return, the risk related to the wallet or have a portfolio without short sale.
the frontend of the application is developed in React JS and relies on two services exposed online: the first, made in Node.JS, provides data on the lessons and progress of the user, the second is written in python and realizes the algorithmic and data retrieval component.